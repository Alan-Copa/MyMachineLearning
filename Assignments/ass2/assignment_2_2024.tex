\documentclass[a4paper,12pt]{article}
\usepackage{amsmath,amsfonts,amssymb}
\usepackage{graphicx}
\usepackage{hyperref}
\usepackage{enumitem}
\usepackage{geometry}
\usepackage{titlesec}

\geometry{margin=1in}

\titleformat{\section}{\large\bfseries}{\thesection}{1em}{}

% Title Section
\title{Machine Learning Assignment 2\\
Neural Networks}
\date{Submission deadline: November 21, 2024}
\author{Alan Copa}

\begin{document}

\maketitle

\noindent Please submit your solution in PDF format (preferably, but not necessarily, \LaTeX --- this .tex file can be found on iCorsi). Handwriting and scanned documents are not allowed. 
In case you need further help, please write on iCorsi or contact me at \href{mailto:vincent.herrmann@idsia.ch}{vincent.herrmann@usi.ch}.

% Problem 1 Section
\section{Problem 1. Calculating Gradients (20 points)}
A two-layer neural network is defined by the following equation: 
$$
y_k := \mathbf{w}^{(2) \top} f\big( W^{(1)} \mathbf{x}_k + \mathbf{b}^{(1)}\big) + b^{(2)}
$$
The values of the input vectors $\mathbf{x}_k \in \mathbb{R}^4$ (summarized in the matrix $X \in \mathbb{R}^{3 \times 4}$), targets $t_k$ (summarized in the vector $\mathbf{t} \in \mathbb{R}^3$), weights $W^{(1)} \in \mathbb{R}^{2 \times 4}$ and $\mathbf{w}^{(2)} \in \mathbb{R}^2$ as well as biases $\mathbf{b}^{(1)} \in \mathbb{R}^2$ and $b^{(2)} \in \mathbb{R}$ are given below. 
\begin{align*}
    X &=
    \begin{bmatrix}
        \text{---} & \mathbf{x}_1 & \text{---} \\
        \text{---} & \mathbf{x}_2 & \text{---} \\
        \text{---} & \mathbf{x}_3 & \text{---}
    \end{bmatrix}
    = [[ 0.2, -0.4, 0.1, 0.5], [ 0.8, 1.0, -0.1, 0.3], [-0.5,  0.9, 0.2, -0.8]] \\
    \mathbf{t} &= 
    \begin{bmatrix}
        t_1 \\
        t_2 \\
        t_3 
    \end{bmatrix}
    = [-0.5, 0.1, 0.8] \\
    W^{(1)} &= [[-0.6, 0.7,  0.9, -0.3], [0.4,  0.5, -1.0, 0.1]] \\
    \mathbf{b}^{(1)} &= [-0.4, -0.5] \\
    \mathbf{w}^{(2)} &= [ 0.2, 0.6] \\
    b^{(2)} &= -0.7 \\
\end{align*}
The function $f$ is the ReLU nonlinearity, applied element-wise. The loss of the model is 
$$
    L := \frac{1}{2} \sum_{k=1}^3(y_k-t_k)^2.
$$
\begin{enumerate}
    \item Numerically calculate the loss and the gradients of $L$ with respect to $W^{(1)}$, $\mathbf{w}^{(2)}$, $\mathbf{b}^{(1)}$ and $b^{(2)}$, and explain your process.
    You may use a calculator or math software (numpy, matlab, ...), but no auto-differentiation libraries like PyTorch or Jax.
    Tip: If you use matrix-matrix multiplications involving $X$, you don't need to do multiple explicit forward and backward passes.
    \item It is more computationally efficient to calculate gradients in a neural network when we start the chain of derivatives on the side of the loss and then going backward, as opposed to starting on the input side and going forward.
    Explain why this is the case using the given example.
\end{enumerate}

\subsection{Solution}
\subsubsection{Loss}
Let's start by computing the ReLU function element-wise: $f(x) = max(0,x)$
$$
ReLU = f\big( W^{(1)} X^T + \mathbf{b}^{(1)}\big)
$$
$$
ReLU = f\big( \begin{bmatrix}
-0.6 & 0.7 & 0.9 & -0.3 \\
0.4 & 0.5 & -1.0 & 0.1
\end{bmatrix}
\begin{bmatrix}
0.2 & 0.8 & -0.5 \\
-0.4 & 1.0 & 0.9 \\
0.1 & -0.1 & 0.2 \\
0.5 & 0.3 & -0.8
\end{bmatrix}
+ \begin{bmatrix} 
-0.4 \\ -0.5 
\end{bmatrix}\big)
$$
$$
ReLU =
f\big(\begin{bmatrix} 
-0.86 & -0.36 & 0.95 \\ 
-0.67 & 0.45 & -0.53 
\end{bmatrix}\big)
$$
$$
ReLU = 
\begin{bmatrix} 
0 & 0 & 0.95 \\ 
0 & 0.45 & 0
\end{bmatrix}
$$
Now let's compute Y, since we are computing vector Y instead of each component $y_k$, $b^{(2)}$ is $\begin{bmatrix} -0.7 & -0.7 -0.7\end{bmatrix}$
$$
Y = \mathbf{w}^{(2) \top} ReLU + b^{(2)}
$$
$$
Y =
\begin{bmatrix} 0.2 & 0.6 \end{bmatrix}
\begin{bmatrix} 
0 & 0 & 0.95 \\ 
0 & 0.45 & 0
\end{bmatrix}
+ \begin{bmatrix} -0.7 & -0.7 -0.7\end{bmatrix}
$$
$$
Y = \begin{bmatrix} -0.7 & -0.43 & -0.51 \end{bmatrix}
$$
Now using the components of Y and t: to solve $L := \frac{1}{2} \sum_{k=1}^3(y_k-t_k)^2.$
$$(y_1 - t_1)^2 = (-0.7 - (-0.5))^2 = (-0.2)^2 = 0.04$$
$$(y_2 - t_2)^2 = (-0.43 - 0.1)^2 = (-0.53)^2 = 0.2809$$
$$(y_3 - t_3)^2 = (-0.51 - 0.8)^2 = (-1.31)^2 = 1.7161$$

$$L = \frac{1}{2} (0.04 + 0.2809 + 1.7161) = \frac{1}{2} \cdot 2.037 = 1.0185$$

\newpage

\subsubsection{Gradients}

$$
\frac{\partial L}{\partial W^{(1)}} = \frac{\partial L}{\partial y_k} \frac{\partial y_k}{\partial f(\cdot)} \frac{\partial f(\cdot)}{\partial W^{(1)}} = \sum_{k=1}^3 (y_k - t_k) \cdot w^{(2)} \cdot \mathbf{1}_{ReLU_k > 0} \cdot x_k^\top
$$
$$
\frac{\partial L}{\partial W^{(1)}} = \big( (\mathbf{w}^{(2)} \cdot \mathbf{1}_{z > 0}) \cdot (Y^T - t) \big) \cdot X
$$
$$
\frac{\partial L}{\partial W^{(1)}} =
\begin{bmatrix}
0.131 & -0.2358& -0.0524 & 0.2096 \\
-0.2544 & -0.318 & 0.0318 & -0.0954
\end{bmatrix}
$$

\vspace{1cm}

$$
\frac{\partial L}{\partial \mathbf{w}^{(2)}} = \frac{\partial L}{\partial y_k} \frac{\partial y_k}{\partial w^{(2)}} =  \sum_{k=1}^3(y_k-t_k) f\big( W^{(1)} x_k + \mathbf{b}^{(1)}\big)
$$
$$
\frac{\partial L}{\partial \mathbf{w}^{(2)}} =
ReLU \cdot (Y^T - t)
$$
$$
\frac{\partial L}{\partial \mathbf{w}^{(2)}} = \begin{bmatrix}
-1.2445 \\
-0.2385
\end{bmatrix}
$$

\vspace{1cm}

$$
\frac{\partial L}{\partial \mathbf{b}^{(1)}} = \frac{\partial L}{\partial y_k} \frac{\partial y_k}{\partial f(\cdot)} \frac{\partial f(\cdot)}{\partial b^{(1)}} = \sum_{k=1}^3 (y_k - t_k) \cdot w^{(2)} \cdot \mathbf{1}_{ReLU_k > 0}
$$
$$
\frac{\partial L}{\partial \mathbf{b}^{(1)}} = \big( \mathbf{w}^{(2)} \cdot \mathbf{1}_{z > 0} \big) \cdot (Y^T - t)
$$
$$
\frac{\partial L}{\partial \mathbf{b}^{(1)}} = \begin{bmatrix}
-0.262 \\
-0.318
\end{bmatrix}
$$

\vspace{1cm}

$$
\frac{\partial L}{\partial b^{(2)}} = \frac{\partial L}{\partial y_k} \frac{\partial y_k}{\partial b^{(2)}} =  \sum_{k=1}^3(y_k-t_k) \cdot 1= -2.04 \cdot 1 = -2.04
$$
Intermediate Results
$$
\frac{\partial L}{\partial y_k} = \sum_{k=1}^3(y_k-t_k) = (-0.2) + (-0.53) + (-1.31) = -2.04
$$
$$
\frac{\partial y_k}{\partial b^{(2)}} = 1
$$
$$
\frac{\partial y_k}{\partial w^{(2)}} = f\big( W^{(1)} x_k + \mathbf{b}^{(1)}\big)
$$
$$
\frac{\partial y_k}{\partial f(\cdot)} = w^{(2)}
$$
$$
\frac{\partial f(\cdot)}{\partial b^{(1)}} = \mathbf{1}_{ReLU_k > 0}
$$
$$
\frac{\partial f(\cdot)}{\partial w^{(1)}} = \mathbf{1}_{ReLU_k > 0} \cdot x_k^T
$$
$$
\mathbf{1}_{ReLU_k > 0} = \begin{bmatrix}
0 & 0 & 1 \\
0 & 1 & 0
\end{bmatrix}
$$
Note:
$
w^{(2)} \cdot \mathbf{1}_{ReLU_k > 0}
$ is done element-wise

\subsection{Solution}

It is more computationally efficient to calculate gradients in a neural network when
we start the chain of derivatives on the side of the loss and then going backward because we can reuse intermediate results (partial derivatives that we already computed) to compute new gradients, saving a lot of computation time and power.

% Problem 2 Section
\section*{Problem 2. Gradient Descent (15 points)}
If we choose a learning rate that is too high, gradient descent can diverge (even in the case of convex functions).
Assume we want to use gradient descent to find the value of $\mathbf{u} = \begin{bmatrix}
    x \\ y
\end{bmatrix}$ that minimizes $f(\mathbf{u}) = \mathbf{u}^T \begin{bmatrix}
    5 & 2 \\
    2 & 2
\end{bmatrix} \mathbf{u}$, or equivalently $f(x, y) = 5 x^2 + 2 y^2 + 4 x y$. The update via gradient descent can be written as $\mathbf{u}_{n+1} = \mathbf{u}_n - \eta \nabla f(\mathbf{u}_n)$, with $\eta$ being the learning rate. 
What is the range of values that $\eta$ can take so that gradient descent eventually converges to the minimum, no matter the starting point? Explain your reasoning–the correct approach is more important than the numerical result.  

\subsection*{Problem 2. Solution}

Since $ f(u) $ can be expressed in a quadratic form $ f(x) = x^T A x $ i.e \\
$$
f(u) = \mathbf{u}^T \begin{bmatrix}
    5 & 2 \\
    2 & 2
\end{bmatrix}
\mathbf{u}
$$
and A is symmetric, we know that the gradient is 
$$
\nabla f(u) = 2Au = 2 \begin{bmatrix}
    5 & 2 \\
    2 & 2
\end{bmatrix}
\mathbf{u}
$$
and the Hessian Matrix is
$$
\nabla^2 f(u) = 2A = 2 \begin{bmatrix}
    5 & 2 \\
    2 & 2
\end{bmatrix}
$$
\subsubsection{Optimal Learning Rate}
We want to converge to the minimum of $f(u)$, to do this we need find the optimal learning rate $\eta$ at step n. Since the goal is to converge at the minimum of $f(u)$ then we can minimize $f$ respect to $u_{n+1}$ and find the optimal learning rate $\eta$ at step n.
$$
\frac{d f(\mathbf{u}_n - \eta \nabla f(\mathbf{u}_n))}{d \eta} = 0
$$
$$
\frac{df}{d\eta}(\mathbf{u}_n - \eta \nabla f(\mathbf{u}_n))^T A (\mathbf{u}_n - \eta \nabla f(\mathbf{u}_n)) = 0
$$
$$
-2 (\nabla f(\mathbf{u}_n))^T A \mathbf{u}_n + 2\eta (\nabla f(\mathbf{u}_n))^T A (\nabla f(\mathbf{u}_n)) = 0
$$
$$
- (\nabla f(\mathbf{u}_n))^T A \mathbf{u}_n + \eta (\nabla f(\mathbf{u}_n))^T A (\nabla f(\mathbf{u}_n)) = 0
$$
$$
\eta = \frac{(\nabla f(\mathbf{u}_n))^T A \mathbf{u}_n}{(\nabla f(\mathbf{u}_n))^T A (\nabla f(\mathbf{u}_n))}
$$
\subsubsection{Learning Rate range of values}

To converge, the update of $f$ must be stable:
$$
u_{n+1} = u_n - \eta \nabla f(u_n)
$$
$$
u_{n+1} = (I - 2\eta A) u_n
$$
For convergence, the eigenvalues of the matrix \( (I - 2\eta A) \) must lie within the unit circle, so we ensure that the learning rate is not too big and the gradient step diminishes and is not amplified.
$$
|1 - 2\eta \lambda_i| < 1 \quad \forall i
$$

1. \( 1 - 2\eta \lambda_i > -1 \):
$$
2\eta \lambda_i < 2 \quad \Rightarrow \quad \eta < \frac{1}{\lambda_i}
$$

2. \( 1 - 2\eta \lambda_i < 1 \):
$$
-2\eta \lambda_i < 0 \quad \Rightarrow \quad \eta > 0
$$

Together are:
$$
0 < \eta < \frac{1}{\lambda_i} \quad \forall i
$$
The biggest eigenvalue will limit the condition:
$$
0 < \eta < \frac{1}{\lambda_{\max}}
$$
To find the eigenvalues of the matrix:
$$
A = \begin{bmatrix} 5 & 2 \\ 2 & 2 \end{bmatrix}
$$
we solve the characteristic equation:
$$
\det(A - \lambda I) = \det\begin{bmatrix} 5 - \lambda & 2 \\ 2 & 2 - \lambda \end{bmatrix} = 0
$$
Compute the determinant:
$$
\det(A - \lambda I) = (5 - \lambda)(2 - \lambda) - 4 = \lambda^2 - 7\lambda + 6
$$
Solve:
$$
\lambda^2 - 7\lambda + 6 = (\lambda - 6)(\lambda - 1) = 0
$$
The eigenvalues are:
$$
\lambda_1 = 6, \quad \lambda_2 = 1
$$
The largest eigenvalue is \( \lambda_{\max} = 6 \), thus the range of the learning rate is:
$$
0 < \eta < \frac{1}{6}
$$

\newpage
% Problem 3 Section
\section*{Problem 3. Convolutional Networks (15 points)}
We start with an input image of size $32 \times 32$ and three channels. 
Provide a sequence of alternating convolutional and pooling layers that transform this image into 64 feature maps of size $4 \times 4$. 

For each convolutional layer, specify the kernel size (e.g., $3 \times 3$, $5 \times 5$ or $7 \times 7$), the number of input channels, and the number of filters. 
All convolutional layers should have a stride of one in either direction. No padding is used. 
All pooling layers should be max-pooling layers with a kernel size of $2 \times 2$ and a stride of two in each direction. 
Give the size and the number of the intermediate feature maps (i.e., the outputs of each layer).
Calculate the total number of learnable parameters in your network (the number of weights in biases in all layers).


Your answer should look roughly like this: 

\noindent Architecture 1:
\begin{itemize}
    \item input feature map: $32 \times 32$, $3$ feature maps
    \item conv layer: kernel size $? \times ?$, $?$ input feature maps, $?$ output feature maps
    \item output conv layer: $? \times ?$, $?$ feature maps
    \item pooling layer: $2 \times 2$
    \item output pooling layer: $? \times ?$, $?$ channels/feature maps
    \item conv layer: ...
    \item ...
    \item output pooling/conv layer: $4 \times 4$, $64$ channels/feature maps
\end{itemize}
Total number of learnable parameters (weights and biases): ? 

\subsection*{Problem 3 Solution}

\subsection*{Formulas}
\begin{itemize}
    \item \textbf{Output size for convolutional layers:}
    \[
    \text{Output size} = \frac{\text{Input size} - \text{Kernel size} + 1}{\text{stride}}
    \]
    \[
    \text{In this exercise we have stride = 1x1}
    \]
    \item \textbf{Number of learnable parameters in a convolutional layer:}
    \[
    \text{Parameters} = \text{Kernel size} \times \text{Input channels} \times \text{Output channels} + \text{Output channels (Biases)}
    \]
    \item \textbf{Output size for pooling layers:}
    \[
    \text{Output size} = \frac{\text{Input size}}{\text{Stride}}
    \]
\end{itemize}

% \subsection*{Architecture}
% \begin{enumerate}
%     \item \textbf{Input feature map:} \( 32 \times 32 \), 3 feature maps.

%     \item \textbf{Conv Layer 1:}
%     \begin{itemize}
%         \item Kernel size: \( 3 \times 3 \)
%         \item Input channels: 3
%         \item Output channels: 16
%         \item Output size:
%         \[
%         \text{Output size} = (32 - 3 + 1) \times (32 - 3 + 1) = 30 \times 30
%         \]
%         \item Output feature maps: \( 30 \times 30 \), 16 feature maps.
%         \item Number of parameters:
%         \[
%         \text{Weights: } 3 \times 3 \times 3 \times 16 = 432, \quad \text{Biases: } 16
%         \]
%         \[
%         \text{Total: } 432 + 16 = 448
%         \]
%     \end{itemize}

%     \item \textbf{Max-Pooling Layer 1:}
%     \begin{itemize}
%         \item Pooling size: \( 2 \times 2 \)
%         \item Stride: 2
%         \item Output size:
%         \[
%         \text{Output size} = \frac{30}{2} \times \frac{30}{2} = 15 \times 15
%         \]
%         \item Output feature maps: \( 15 \times 15 \), 16 feature maps.
%     \end{itemize}

%     \item \textbf{Conv Layer 2:}
%     \begin{itemize}
%         \item Kernel size: \( 3 \times 3 \)
%         \item Input channels: 16
%         \item Output channels: 32
%         \item Output size:
%         \[
%         \text{Output size} = (15 - 3 + 1) \times (15 - 3 + 1) = 13 \times 13
%         \]
%         \item Output feature maps: \( 13 \times 13 \), 32 feature maps.
%         \item Number of parameters:
%         \[
%         \text{Weights: } 3 \times 3 \times 16 \times 32 = 4608, \quad \text{Biases: } 32
%         \]
%         \[
%         \text{Total: } 4608 + 32 = 4640
%         \]
%     \end{itemize}

%     \item \textbf{Max-Pooling Layer 2:}
%     \begin{itemize}
%         \item Pooling size: \( 2 \times 2 \)
%         \item Stride: 2
%         \item Output size:
%         \[
%         \text{Output size} = \frac{13}{2} \times \frac{13}{2} = 6 \times 6
%         \]
%         \item Output feature maps: \( 6 \times 6 \), 32 feature maps.
%     \end{itemize}

%     \item \textbf{Conv Layer 3:}
%     \begin{itemize}
%         \item Kernel size: \( 3 \times 3 \)
%         \item Input channels: 32
%         \item Output channels: 64
%         \item Output size:
%         \[
%         \text{Output size} = (6 - 3 + 1) \times (6 - 3 + 1) = 4 \times 4
%         \]
%         \item Output feature maps: \( 4 \times 4 \), 64 feature maps.
%         \item Number of parameters:
%         \[
%         \text{Weights: } 3 \times 3 \times 32 \times 64 = 18432, \quad \text{Biases: } 64
%         \]
%         \[
%         \text{Total: } 18432 + 64 = 18496
%         \]
%     \end{itemize}
% \end{enumerate}

% \subsection*{Final Architecture Summary}
% \begin{itemize}
%     \item Input feature map: \( 32 \times 32 \), 3 feature maps.
%     \item Conv Layer 1: \( 3 \times 3 \) kernel, 3 input channels, 16 output channels.
%         \begin{itemize}
%             \item Output: \( 30 \times 30 \), 16 feature maps.
%         \end{itemize}
%     \item Max-Pooling Layer 1: \( 2 \times 2 \) pooling.
%         \begin{itemize}
%             \item Output: \( 15 \times 15 \), 16 feature maps.
%         \end{itemize}
%     \item Conv Layer 2: \( 3 \times 3 \) kernel, 16 input channels, 32 output channels.
%         \begin{itemize}
%             \item Output: \( 13 \times 13 \), 32 feature maps.
%         \end{itemize}
%     \item Max-Pooling Layer 2: \( 2 \times 2 \) pooling.
%         \begin{itemize}
%             \item Output: \( 6 \times 6 \), 32 feature maps.
%         \end{itemize}
%     \item Conv Layer 3: \( 3 \times 3 \) kernel, 32 input channels, 64 output channels.
%         \begin{itemize}
%             \item Output: \( 4 \times 4 \), 64 feature maps.
%         \end{itemize}
% \end{itemize}

% \subsection*{Total Number of Learnable Parameters}
% \begin{itemize}
%     \item Conv Layer 1: \( 448 \)
%     \item Conv Layer 2: \( 4640 \)
%     \item Conv Layer 3: \( 18496 \)
% \end{itemize}
% \[
% \text{Total Parameters} = 448 + 4640 + 18496 = 23584
% \]


\subsection*{Architecture}
\begin{itemize}
    \item \textbf{Input feature map:} \( 32 \times 32 \), 3 feature maps.

    \item \textbf{Conv Layer 1:}
    \begin{itemize}
        \item Kernel size: \( 5 \times 5 \)
        \item Input channels: 3
        \item Output channels: 8
        \item Output size:
        \[
        \text{Output size} = (32 - 5 + 1) \times (32 - 5 + 1) = 28 \times 28
        \]
        \item Output feature maps: \( 28 \times 28 \), 8 feature maps.
        \item Number of parameters:
        \[
        \text{Weights: } 5 \times 5 \times 3 \times 8 = 600, \quad \text{Biases: } 8
        \]
        \[
        \text{Total: } 600 + 8 = 608
        \]
    \end{itemize}

    \item \textbf{Max-Pooling Layer 1:}
    \begin{itemize}
        \item Pooling size: \( 2 \times 2 \)
        \item Stride: 2
        \item Output size:
        \[
        \text{Output size} = \frac{28}{2} \times \frac{28}{2} = 14 \times 14
        \]
        \item Output feature maps: \( 14 \times 14 \), 8 feature maps.
    \end{itemize}

    \item \textbf{Conv Layer 2:}
    \begin{itemize}
        \item Kernel size: \( 5 \times 5 \)
        \item Input channels: 8
        \item Output channels: 32
        \item Output size:
        \[
        \text{Output size} = (14 - 5 + 1) \times (14 - 5 + 1) = 10 \times 10
        \]
        \item Output feature maps: \( 10 \times 10 \), 32 feature maps.
        \item Number of parameters:
        \[
        \text{Weights: } 5 \times 5 \times 8 \times 32 = 6400, \quad \text{Biases: } 32
        \]
        \[
        \text{Total: } 6400 + 32 = 6432
        \]
    \end{itemize}

    \item \textbf{Max-Pooling Layer 2:}
    \begin{itemize}
        \item Pooling size: \( 2 \times 2 \)
        \item Stride: 2
        \item Output size:
        \[
        \text{Output size} = \frac{10}{2} \times \frac{10}{2} = 5 \times 5
        \]
        \item Output feature maps: \( 5 \times 5 \), 32 feature maps.
    \end{itemize}

    \item \textbf{Conv Layer 3:}
    \begin{itemize}
        \item Kernel size: \( 2 \times 2 \)
        \item Input channels: 32
        \item Output channels: 64
        \item Output size:
        \[
        \text{Output size} = (5 - 2 + 1) \times (5 - 2 + 1) = 4 \times 4
        \]
        \item Output feature maps: \( 4 \times 4 \), 64 feature maps.
        \item Number of parameters:
        \[
        \text{Weights: } 2 \times 2 \times 32 \times 64 = 8192, \quad \text{Biases: } 64
        \]
        \[
        \text{Total: } 8192 + 64 = 8256
        \]
    \end{itemize}
\end{itemize}

\subsection*{Final Architecture}
\begin{itemize}
    \item Input feature map: \( 32 \times 32 \), 3 feature maps.
    \item Conv Layer 1: \( 5 \times 5 \) kernel, 3 input channels, 8 output channels.
        \begin{itemize}
            \item Output: \( 28 \times 28 \), 8 feature maps.
        \end{itemize}
    \item Max-Pooling Layer 1: \( 2 \times 2 \) pooling.
        \begin{itemize}
            \item Output: \( 14 \times 14 \), 8 feature maps.
        \end{itemize}
    \item Conv Layer 2: \( 5 \times 5 \) kernel, 8 input channels, 32 output channels.
        \begin{itemize}
            \item Output: \( 10 \times 10 \), 32 feature maps.
        \end{itemize}
    \item Max-Pooling Layer 2: \( 2 \times 2 \) pooling.
        \begin{itemize}
            \item Output: \( 5 \times 5 \), 32 feature maps.
        \end{itemize}
    \item Conv Layer 3: \( 2 \times 2 \) kernel, 32 input channels, 64 output channels.
        \begin{itemize}
            \item Output: \( 4 \times 4 \), 64 feature maps.
        \end{itemize}
\end{itemize}

\subsection*{Number of Learnable Parameters}
\begin{itemize}
    \item \textbf{Conv Layer 1:} 608 
    \item \textbf{Conv Layer 2:} 6432
    \item \textbf{Conv Layer 3:} 8256
\end{itemize}
\[
\text{Total Parameters} = 608 + 6432 + 8256 = 15,296
\]


\end{document}